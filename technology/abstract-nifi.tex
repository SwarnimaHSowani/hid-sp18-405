\section{Apache NiFi}
\index{Apache NiFi}

Apache NiFi, which is short for NiagaraFiles, is a open source
software project from the Apache Software Foundation designed to
automate the flow of data between software
systems~\cite{hid-sp18-405-wiki-nifi}. Based on the ``NiagaraFiles''
software previously developed by the NSA, Apache NiFi is part of its
technology transfer program in 2014~\cite{hid-sp18-405-wiki-nifi}.
NiFi executes within a Java Virtual Machine with the following primary
components: Web Server, Flow Controller, Extensions, FlowFile
Repository, Content Repository and Provenance Repository. Since NiFi's
fundamental design concepts are closely related to Flow Based
Programming (FBP), some of the above components can be mapped closely
to FBP terms. For example, Flow Controller and FlowFile can be related
to Scheduler and Information Packet in FBP terms
respectively~\cite{hid-sp18-405-wwwoverview-nifi}~\cite{hid-sp18-405-wikifbp-nifi}.
Apache NiFi supports scalable directed graphs of data routing,
transformation, and system mediation logic, aiming at leveraging the
capabilities of the underlying host system on which it is operating,
especially with regard to CPU and
disk~\cite{hid-sp18-405-wwwoverview-nifi}. Some of the high-level
capabilities and objectives of Apache NiFi include: Web-based user
interface, Highly configurable, Data Provenance, Designed for
extension and Secure~\cite{hid-sp18-405-www-nifi}.
